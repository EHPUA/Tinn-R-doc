
\section{Versions released in 2006 (64)}
\subsection*{1.19.1.1 (Dec/27/2006)}
\begin{itemize}
  \item Bug(s) fixed:
    \begin{itemize}
      \item A bug related to version '1.19.1.0' and 'midas.dll' was fixed:
        the content of this dll was compiled inside the binary code to avoid
        its redistribution.
    \end{itemize}
\end{itemize}


\subsection*{1.19.1.0 (Dec/26/2006)}
\begin{itemize}
  \item Parts of the source code were optimized.
  \item The DBase database (RCard and RTip) was upgraded to the native XML engine
    provided by Borland. It is very small, stable and fast.
    \begin{itemize}
      \item Therefore, do not restore the database from old versions to the new
        ones.
      \item It will be easier to integrate \RR{} and Tinn-R in the future.
    \end{itemize}
\end{itemize}


\subsection*{1.19.0.18 (Dec/13/2006)}
\begin{itemize}
  \item The \textit{Print} resource was a bit reworked: it will always close
    the print dialogue after pressing the \textit{Print} option.
  \item The \textit{Application options} was a bit reworked.
  \item Two non-configurable shortcuts were added:
    \begin{enumerate}
      \item \texttt{CTRL + (}: insert (or replace) ($|$)
      \item \texttt{CTRL + )}: insert (or replace) ()
    \end{enumerate}
  \item A new folder named \textit{custom} was added to the program folders.
    It contains two customizable files: \texttt{Rcompletion.r} and
    \texttt{Rconfigure.r}:
    \begin{enumerate}
      \item \texttt{Rcompletion.r}: allows the user to customize the new
        \textit{auto completion} resource (described below);
      \item \texttt{Rconfigure.r}: allows the user to customize the \RR{}
        configuration.
    \end{enumerate}
  \item The first approach of \textit{auto completion} was implemented and
    the default keystroke is \texttt{CTRL + J}, but the user can customize it
    in the \textit{Editor} options: \textit{Options/Main/Editor/Keystrokes}.
    Use it, after any valid word, by pressing the keystroke. For example:

    \begin{footnotesize}
      \begin{verbatim}

        if<CTRL + J> to obtain:
        if (| < )

        ifc<CTRL + J> to obtain:
        if (| < ) {

        }

        fo<CTRL + J> to obtain:
        for (i in 1:i|)


        foc<CTRL + J> to obtain:
        for (i in 1:|) {

        }

        sw<CTRL + J> to obtain:
        switch(|,
        a = ' ',
        b = ' ',
        )

        wh<CTRL + J> to obtain:
        i = 0
        while (i < |) {

          i = i + 1
        }

        eq<CTRL + J> to obtain:
        \begin{equation}\label{eq_01}
          |
        \end{equation}
      \end{verbatim}
    \end{footnotesize}

    Observations:

    \begin{enumerate}
      \item Only two letters were used to define simple structures
        (for example: \textit{fo} = \textit{for}, \textit{fu} = \
        textit{function});
      \item Considering the above, letter \texttt{c} was added
        to complete or more complex structures (for example:
        \textit{foc}, \textit{fuc});
      \item If there are no conflict among the names in the
        \textit{Rcompletion file}, the first letter is enough. For
        example: \textit{s} = \textit{swc}. If there are conflicts, it
        will be necessary to add more letters to complete the diferentiation;
      \item The \texttt{$|$} symbol is used to define where the cursor
        will first stop after the auto completion;
      \item To better understand it, have a look at the file
        \texttt{Rcompletion.r} within the folder \textit{custom}
        where the program was installed and do your own tests.
    \end{enumerate}
\end{itemize}


\subsection*{1.19.0.17 (Dec/10/2006)}
\begin{itemize}
  \item Bug(s) fixed:
    \begin{itemize}
      \item When both options \textit{Organize screen} and
        \textit{Return focus} were marked, it triggered a bug,
        which has been fixed.
    \end{itemize}
\end{itemize}


\subsection*{1.19.0.16 (Dec/08/2006)}
\begin{itemize}
  \item The \textit{Main menu} was reworked and several options now
    have a better logic place.
\end{itemize}


\subsection*{1.19.0.15 (Dec/07/2006)}
\begin{itemize}
  \item The \textit{Main menu} was reworked and several
    options now have a better logic place.
\end{itemize}


\subsection*{1.19.0.14 (Dec/06/2006)}
\begin{itemize}
  \item The interfaces \textit{Search text} and \textit{Replace text} were reworked.
  \item Parts of the source code were optimized.
  \item It is now possible to increase and decrease (non-permanent)
    the font size of any individual file easily:

    \begin{footnotesize}
      \begin{verbatim}
        <SHIFT + CTRL + +> (plus  (+) on the numeric keypad): INCREASE (upper limit is 50)
        <SHIFT + CTRL + -> (minus (-) on the numeric keypad): DECREASE (lower limit is  1)
      \end{verbatim}
    \end{footnotesize}

\end{itemize}


\subsection*{1.19.0.13 (Dec/04/2006)}
\begin{itemize}
  \item All images of the documentation were updated to the latest version.
\end{itemize}


\subsection*{1.19.0.12 (Dec/04/2006)}
\begin{itemize}
  \item Parts of the source code were optimized.
  \item The way as the opened files are closed was enhanced. Now, if any file
    was changed and the user chooses \textit{Cancel}, the current procedure
    will be canceled too. For example, closing: projects, all right, all left,
    etc.
\end{itemize}


\subsection*{1.19.0.11 (Nov/30/2006)}
\begin{itemize}
  \item The \textit{Application options} was a bit reworked. Now the user can
    set the initialization parameters of Rgui. Set it at \textit{Application
      options/R/General/Rgui (Parameters and path)}.
\end{itemize}


\subsection*{1.19.0.10 (Nov/30/2006)}
\begin{itemize}
  \item The \textit{Project} interface was a bit enhanced.
  \item Parts of the source code were optimized.
\end{itemize}


\subsection*{1.19.0.9 (Nov/29/2006)}
\begin{itemize}
  \item Bug(s) fixed:
    \begin{itemize}
      \item A bug related to the \textit{Project} and associated with the prior
        version of Tinn-R (1.19.0.8) was fixed.
    \end{itemize}
\end{itemize}


\subsection*{1.19.0.8 (Nov/28/2006)}
\begin{itemize}
  \item The pop-up menus related to the \textit{Project}, and to the
    \textit{Editor}, were a bit reworked with respect to \textit{Put
      full path in the clipboard} option.
  \item Parts of the source code were optimized.
  \item \textit{Search text} and \textit{Replace text} interface were
    reworked.
\end{itemize}


\subsection*{1.19.0.7 (Nov/27/2006)}
\begin{itemize}
  \item If the variable \textit{rGuiPreferred} is empty (as in the first use)
    in the \textit{ini file}, Tinn-R will check in the Windows registry at
    \textit{HKLM$\backslash$R-core$\backslash$R} to find the path where
    Rgui.exe was installed. So, if the user want to use \texttt{Tinn-R with
      \RR{} console}, it is no longer necessary to set it in the
    \textit{Application options}.
\end{itemize}


\subsection*{1.19.0.6 (Nov/23/2006)}
\begin{itemize}
  \item The command \textit{Send line} to \RR{} was enhanced. Now, whenever
    the user set it not to send comment lines, and the current line is either
    commented or empty, Tinn-R will search below towards the end of file, for
    the first not commented or empty line and will send it automatically.
    Also, it will always search below towards the end of the file, for the first
    not commented or empty line. So, the user's work when
    sending line by line will be reduced.
\end{itemize}


\subsection*{1.19.0.5 (Nov/21/2006)}
\begin{itemize}
  \item The \textit{Application options} was a bit reworked. Now the user can
    decide whether the position of the Tinn-R and Rgui windows will be organized
    automatically or not, when Rgui is running. Set it at \textit{Application
      options/R/Organize}.
\end{itemize}


\subsection*{1.19.0.4 (Nov/18/2006)}
\begin{itemize}
  \item Bug(s) fixed:
    \begin{itemize}
      \item A bug associated with the use of \texttt{CTRL +} to insert content of
        the tip with 'R complex' and another was fixed.
    \end{itemize}
\end{itemize}


\subsection*{1.19.0.3 (Nov/17/2006)}
\begin{itemize}
  \item Now Tinn-R has four multi-highlighters: \textit{HTML complex},
    \textit{PHP complex}, \textit{R complex} and \textit{Sweave}:


    \begin{footnotesize}
      \begin{verbatim}
        1. HTML complex = HTML & JavaScript
        2. PHP complex = HTML & JavaScript & PHP
        3. R complex  = R & URI ('<<<' begin URI '>>>'   end URI)
        4. Sweave = TeX & R ('>>=' begin R '@' end R)

        URI : Uniform Resource Identifiers.
        R: complex: The main syntax is R, '<<<' and '>>>' the tags enable the user
        to insert a block of URI syntax.
        Sweave: The main syntax is TeX, '>>=' and '@' the tags enable the user
        to insert a block of R syntax.
      \end{verbatim}
    \end{footnotesize}

    \begin{footnotesize}
      \begin{verbatim}
        The SynHighlighterURI unit implements an URI syntax highlighter for SynEdit.

        Recognition of URIs is based on the information provided in the document
        "Uniform Resource Identifiers (URI): Generic Syntax" of "The Internet Society",
        that can be found at http://www.ietf.org/rfc/rfc2396.txt.

        Also interesting is http://www.freesoft.org/CIE/RFC/1738/33.htm which describes
        general URL syntax and major protocols.

        These protocols are recognized:
        -------------------------------
        http://
        https://
        ftp://
        mailto:
        news: or news://
        nntp://
        telnet://
        gopher://
        prospero://
        wais://

        As well as commonly used shorthands:
        ------------------------------------
        someone@somewhere.org
        www.host.org

      \end{verbatim}
    \end{footnotesize}

    Example of \RR{} complex:

    \begin{Scode}
      # <<< http://www.r-project.org/ >>>
      # <<< joseclaudio.faria@terra.com.br >>>
      # <<< phgrosjean@sciviews.org >>>

      mean(rnorm(100))
    \end{Scode}

    or

    \begin{Scode}
      # <<< http://www.r-project.org/
      #     joseclaudio.faria@terra.com.br
      #     phgrosjean@sciviews.org >>>

      mean(rnorm(100))
    \end{Scode}

    Therefore, by pressing \texttt{CTRL} and right button of the mouse on any
    valid link, it is possible to follow the link (browser, client of email, etc).
  \item Menu \textit{Options} was a bit reworked to enable the user to set
    \textit{R} or \textit{R complex} as the default syntax.
\end{itemize}


\subsection*{1.19.0.2 (Nov/09/2006)}
\begin{itemize}
  \item The option \texttt{Automatically configure \RR{} (Rprofile.site)}
    was enhanced. Now the user can now decide whether the changes will be
    \texttt{temporary} (on the current session only) or \texttt{permanent}
    (editing automatically the Rprofile.site file). For both, if necessary,
    the user can edit (customize) the file \textit{Rconfigure.r} in the
    folder \textit{doc/English} where Tinn-R is installed. So we think this
    new resource is now more powerful and flexible.
  \item The \textit{Application options} was reworked. Now it is possible to
    set the position of the Tinn-R window (top, bottom, left or right with
    respect to the Rgui). \texttt{Left or right positions are recommended only for very
      large screens}. Set it at \textit{Application options/R/Organize}.
\end{itemize}


\subsection*{1.19.0.1 (Nov/07/2006)}
\begin{itemize}
  \item One new option \texttt{Automatically configure \RR{} (Rprofile.site)}
    was added to the \textit{R} main menu. It will enable the user to
    automatically configure the Rprofile.site file with all packages which
    are necessary to work efficiently. If any of those are not found, \RR{}
    will download and install them when starting.
\end{itemize}


\subsection*{1.19.0.0 (Nov/06/2006)}
\begin{itemize}
  \item Bug(s) fixed:
    \begin{itemize}
      \item Bugs when opening files in Tinn-R by double clicking from
        \textit{Windows Explorer} were fixed. In other words, the
        interaction with \textit{Windows Explorer} was enhanced.
      \item A bug associated with \texttt{SHIFT+ESPACE} was fixed.
    \end{itemize}
  \item Tinn-R has now a new logo/image identity which reflects
    its natural evolution.
  \item The sources codes were deeply remade and optimized.
\end{itemize}


\subsection*{1.18.7.3 (Oct/30/2006)}
\begin{itemize}
  \item Bug(s) fixed:
    \begin{itemize}
      \item A small bug was fixed: from now on, whenever the user starts
        \texttt{R} inside \texttt{Tinn-R}, it will open automatically
        in the \texttt{SDI mode}, otherwise, in the default mode defined
        in the \textit{Rconsole} configuration file. Also, if any file
        happens to be open, \texttt{and active}, the \texttt{R Working
          Directory} will be set to its path. Otherwise, it will be set to
        the default path (../bin) where \RR{} was installed.
      \item Bugs with TCP/IP protocol (associated with the prior version
        1.18.7.2) were fixed: sorry for this. By the way, we would like
        to remember the users there are still some unsolved problems
        with \textit{R output} under this protocol. For example, the
        codes below:

        \begin{Scode}
          f <- function(x) {
            x/10 + 1
          }
        \end{Scode}

        or, i.e, \texttt{any instruction with more than one single line},

        \begin{footnotesize}
          \begin{verbatim}
            a <- 0
            for (i in 1:20) {
              a = i
              if(a <= 5 ) {
                cat('a = ', a, '(<= 5)'); cat('\n')
                next
              }
              if(a == 18) {
                cat('a = ', a, '(= 18)'); cat('\n')
                break
              }
            }
          \end{verbatim}
        \end{footnotesize}

        will always generate errors. We've been working hard in order to solve it,
        but it is not easy!
        Moreover, consider that this protocol's main objective is to be used with \RR{}
        explorer interface and \texttt{any other use is still experimental}.
    \end{itemize}
\end{itemize}


\subsection*{1.18.7.2 (Oct/30/2006)}
\begin{itemize}
  \item Bug(s) fixed:
    \begin{itemize}
      \item A small bug, when closing the \textit{Rgui} and the choice
        of \textit{Cancel}, was fixed.
    \end{itemize}
  \item The \textit{Application options} was reworked a little bit. The
    options enabling the user to set Tinn-R to always show \textit{Tips}
    and \textit{Completion} were removed. We think it was an annoyance.
    Now, the triggers can be used to do that.
  \item One new function was added both in the \textit{R toolbar} and the
    \textit{R} menu: \textit{Toogle start/close}. It will enable the user
    to start/close the preferred user \textit{Rgui}.
  \item From now on, whenever the user starts \texttt{R} inside
    \texttt{Tinn-R}, it will open automatically in the \texttt{SDI mode},
    otherwise, in the default mode defined in the \textit{Rconsole}
    configuration file. Also, if there is any opened file, the
    \texttt{R Working Directory} will be set to its path. Otherwise,
    it will be set to the default path (../bin) where \RR{} was installed.
  \item Several parts of the source code were optimized.
  \item The performance of the communication with the \textit{R server}
    (under all protocols) was enhanced. Now the user can to try to set
    the delay for computational synchronization
    (\textit{Options/Main/Application/R/R server}) to small values
    (50, 80 or 100 ms, in harmony with the hardware). We recommend
    testing different delays until a good performance is obtained.
  \item Icons of \textit{Controlling R} were remade.
\end{itemize}


\subsection*{1.18.7.1 (Oct/26/2006)}
\begin{itemize}
  \item Many cosmetic changes were made to the program's interface
    and documentation.
\end{itemize}


\subsection*{1.18.7.0 (Oct/25/2006)}
\begin{itemize}
  \item All the HTML documentation of Tinn-R was easily remade; we became
    very tired of the \textit{blue colour}:
    \htmladdnormallink{txt2tags}{http://txt2tags.sourceforge.net/}
    is an exceptional tool!
  \item From now on (Enio Jelihovschi - a new Tinn-R member) is 
    responsible for the production and quality of all Tinn-R documentation.
    Welcome Enio, we wish you have a good work on board.
  \item One new function was added both to the \RR{} toolbar and the \RR{}
    menu: \textit{TCP/IP toggle connetion}.
\end{itemize}


\subsection*{1.18.6.13 (Oct/24/2006)}
\begin{itemize}
  \item Parts of the source code were optimized.
  \item \textit{Tools/Sort} was enhanced: it now enables you to sort by
    \textit{Strings}, \textit{Numbers} and \textit{Dates}.
  \item Popup menu associated with tabs was a bit reworked.
  \item The speller now will advise whenever it finishes the
    corrections.
\end{itemize}


\subsection*{1.18.6.12 (Oct/22/2006)}
\begin{itemize}
  \item Several parts of the source code were optimized.
  \item Popup menu associated with the \textit{Tabs} was enhanced. It now
    additionally enables you to close all files located at the \texttt{right}
    or \texttt{left} from the active tab.
  \item Main menu \textit{File} was a bit remade.
  \item \textit{Popup menu editor} was a bit remade.
\end{itemize}


\subsection*{1.18.6.11 (Oct/20/2006)}
\begin{itemize}
  \item Bug(s) fixed:
    \begin{itemize}
      \item A small bug associated with main menu, \textit{Tools} was fixed.
    \end{itemize}
\end{itemize}


\subsection*{1.18.6.10 (Oct/20/2006)}
\begin{itemize}
  \item Several parts of the source code were optimized.
  \item \RR{} explorer performance was enhanced. Now you can to try to set
    the delay for computational synchronization
    (\textit{Options/Main/Application/R/R server}) to small values
    (50, 80 or 100 ms, in harmony with your hardware). We recommend
    testing different delays until you obtain a good performance.
  \item We also recommend you to use the resources of \textit{R explorer}
    under TCP/IP protocol (it makes the Rgui console cleaner). By the way,
    there are still some unsolved problems with \textit{R output} under
    this protocol.
  \item Window \textit{About} was remade.
\end{itemize}


\subsection*{1.18.6.9 (Oct/17/2006)}
\begin{itemize}
  \item Bug(s) fixed:
    \begin{itemize}
      \item A small bug associated with \texttt{CTRL + W} shortcut
        closing two files instead of just one, was fixed.
    \end{itemize}
  \item The \textit{Application options} was somehow reworked.
    When closing it, the screen will not be organized. Therefore it will
    be necessary to toggle it on the main menu \textit{View/Organise screen}
    or in the toolbar \textit{Tools}.
  \item The options \textit{Send selection} will be enabled only if any selection
    was done.
\end{itemize}


\subsection*{1.18.6.8 (Oct/10/2006)}
\begin{itemize}
  \item Bug(s) fixed:
    \begin{itemize}
      \item A small bug associated with \textit{Status bar} was fixed.
    \end{itemize}
  \item The \textit{Application options} was reworked. It is now possible
    to set the position of \RR{} (top or bottom) on \textit{Organize screen}
    resource by using \textit{Application options/R/General/settings}.
  \item Clipboard actions were enhanced.
\end{itemize}


\subsection*{1.18.6.7 (Oct/08/2006)}
\begin{itemize}
  \item A new option \textit{Organize screen} was added to the menu
    \textit{View} and the button associated to it was added to the
    bar \textit{Tools}. This has been, up to now, an old user's request,
    and it enables you to put, in a fast way, the \texttt{R Console on
      the top} of the screen and the \texttt{Tinn-R on the bottom}.
  \item \textit{Tools/Sort} was enhanced: now it maintains the file
    in the original \textit{Top line} position.
  \item The way project files (*.tps) are opened was changed: the
    textual file (by default) will not be opened anymore. The user,
    now, can easily edit the file in the \textit{Project} interface.
\end{itemize}


\subsection*{1.18.6.6 (Oct/06/2006)}
\begin{itemize}
  \item A new option \textit{Sort} was added to menu \textit{Tools}.
    This enables the user to sort an entire file (any selection mode)
    or selection (for while only for \textit{smNormal} and
    \textit{smColumn}).
  \item \textit{Popup menu editor} was reworked.
  \item Icons of \textit{R explorer} were changed.
\end{itemize}


\subsection*{1.18.6.5 (Oct/03/2006)}
\begin{itemize}
  \item Bug(s) fixed:
    \begin{itemize}
      \item A bug related to \textit{Spell}, that always broke down when
        finding \texttt{$<$} or \texttt{$<$-} symbol, was fixed. The
        origin of this bug was a component conflict (hard coded) among \
        the HTML brackets, declared as below:
      \item A bug associated to \textit{Project/Close entire project} was fixed.

        \begin{footnotesize}
          \begin{verbatim}
            const
            OpenBracket: array[THTMLBracket] of PChar=('<', '<!--', '<%');
            CloseBracket: array[THTMLBracket] of PChar=('>', '-->', '%>');
          \end{verbatim}
        \end{footnotesize}

        and the \RR{} assign symbols:

        \begin{footnotesize}
          \begin{verbatim}
            <-
            <<-
            ->
            ->>
          \end{verbatim}
        \end{footnotesize}

        Therefore, if any file has HTML syntax (with any of those tags above) all
        text among the declared brackets is free of speller. We believe that will
        work nicely from now on.
    \end{itemize}
\end{itemize}


\subsection*{1.18.6.4 (Oct/01/2006)}
\begin{itemize}
  \item The \textit{Shortcuts customization} interface was reworked.
\end{itemize}


\subsection*{1.18.6.3 (Sep/29/2006)}
\begin{itemize}
  \item The \textit{Shortcuts customization} interface was remade, we
    think it is now more user friendly.
  \item There is a new button in the interface above, that enables the
    user to restore the main default Tinn-R shortcuts.
\end{itemize}


\subsection*{1.18.6.2 (Sep/28/2006)}
\begin{itemize}
  \item Bug(s) fixed:
    \begin{itemize}
      \item A small bug associated with the buttons icons of \textit{Tools/R explorer}
        related with version 1.18.6.1 was fixed.
    \end{itemize}
\end{itemize}


\subsection*{1.18.6.1 (Sep/28/2006)}
\begin{itemize}
  \item Version 1.18.6.0 was updated, due to some problems detected in
    file \textit{default.bin}. Accordingly, we \texttt{strongly recommended}
    that, before using this new version, the user \texttt{manually} removes
    that file. In order to help the user in finding out where Tinn-R stores all
    the \texttt{ini files}, a new option was added to the main menu
    \textit{Tools} named \textit{Ini files}. The \textit{default.bin} file
    is located in the sub-folder \textit{shortcuts}. Therefore, go to the
    folder and delete it. Sorry for the inconvenience.
\end{itemize}


\subsection*{1.18.6.0 (Sep/26/2006)}
\begin{itemize}
  \item Several parts of the source code were optimized.
  \item Menu and pop-up menu options were remade and some icons replaced/changed.
  \item A new option \textit{Shortcuts customization} was added to the menu
    \textit{Options}. This will enable the user to manage (edit, load and save)
    all actions of Tinn-R interface. The latest in use will be always reloaded
    when restarting the program.
\end{itemize}


\subsection*{1.18.5.12 (Sep/15/2006)}
\begin{itemize}
  \item Bug(s) fixed:
    \begin{itemize}
      \item A few small bugs pointed out by users were fixed.
    \end{itemize}
\end{itemize}


\subsection*{1.18.5.12 (Sep/13/2006)}
\begin{itemize}
  \item Bug(s) fixed:
    \begin{itemize}
      \item A small bug, which has been happening whenever the user opens the first file
        located in the floppy drive and access this floppy in subsequent files, was fixed.
    \end{itemize}
\end{itemize}


\subsection*{1.18.5.11 (Sep/12/2006)}
\begin{itemize}
  \item Bug(s) fixed:
    \begin{itemize}
      \item A small bug, related to versions 1.18.5.9 and 1.18.5.10, associated with
        menu \textit{File/Save as}, was fixed.
    \end{itemize}
\end{itemize}


\subsection*{1.18.5.10 (Sep/11/2006)}
\begin{itemize}
  \item The command \textit{Send line} to \RR{} was enhanced. Now, whenever
    the user sets it not to send comment lines, and the current is either
    commented or empty, Tinn-R will search below to the end of file, for
    the first not commented or empty line.
\end{itemize}


\subsection*{1.18.5.9 (Sep/10/2006)}
\begin{itemize}
  \item Bug(s) fixed:
    \begin{itemize}
      \item A small bug associated with dragging files in project interface
        was fixed. Before, whenever you changed files among groups it
        would not be associated with changes.
    \end{itemize}
  \item One new function was added both to the \RR{} toolbar and the \RR{}
    menu: \textit{Set work directory (current file path)}.
  \item The icons associated with \textit{Computer} interface (\textit{Add}
    and \textit{Remove} favorite folder) were replaced.
  \item Tinn-R now recognizes all valid extensions for any syntax when
    saving files. For example, the new valid \RR{} extensions are: *.r,
    *.Rhistory, *.q, *.s and *.ssc; so, whenever you type:
    \begin{itemize}
      \item \texttt{Myfile} - it will be saved as \texttt{Myfile.r} (the
        first extension is the default);
      \item \texttt{Myfile.R} - it will be saved as \texttt{Myfile.R}
        (\texttt{R} is a valid extension and Tinn-R is now case insensitive
        for that purpose);
      \item \texttt{Myfile.Rhistory} - it will be saved as
        \texttt{Myfile.Rhistory} (Rhistory is a valid extension);
      \item \texttt{Myfile.help} - it will be saved as \texttt{Myfile.help.r}
        (\texttt{help} it is not a valid extension and *.r is the default);
      \item PS: in order to save any file with any extension, you must choose
        \textit{All} syntax.
    \end{itemize}
  \item Whenever the \textit{tab file} has the focus, a new associated
    pop-up menu is available with the following options: \textit{Close},
    \textit{Close others} and \textit{Close all}.
  \item The way Tinn-R opens and closes a project, was changed. Now, any *.tps
    (from any source) will also be opened in the \textit{Project} interface
    and all files of the project will be closed only and only if you choose
    the option \textit{Close entire project}.
  \item The \textit{Project} interface was isomehow enhanced. Now, the user
    can type \texttt{DELETE} to delete either groups or files, \texttt{INSERT}
    to add files to selected groups and $<$\textbf{CTRL + INSERT}$>$ to add
    the current file to the selected group.
\end{itemize}


\subsection*{1.18.5.8 (Ago/03/2006)}
\begin{itemize}
  \item Tinn-R version 1.18.5.7 was pre-released (restricted to developers
    only).
  \item The conflict between the \textit{Editor option/Alt sets column mode}
    and the possible user option \textit{Format/Selection mode/Column}
    was fixed.
\end{itemize}


\subsection*{1.18.5.7 (Ago/01/2006)}
\begin{itemize}
  \item A first approximation to build a speller in Tinn-R was made.
  \item To install this new resource do the following:
    \begin{enumerate}
      \item Close Tinn-R;
      \item Download the dictionaries\\
        {http://www.luziusschneider.com/Speller/English/index.htm}
        that you want and install it from the installer (for
        example ISpEnFrGe.exe).
    \end{enumerate}
  \item It is very simple, works nice, has power and all resources are
    opensource.
  \item For a while this component did not recognize the main SynEdit
    component used for edition. Consequently, it was necessary to do the spelling
    in two steps, making the correction in a new page (\textit{Spell})
    on \textit{Results} interface.
  \item \texttt{Whenever it finds any selection}, the correction will be
    made only for it, as well as for subsequent updates (or for subsequent update).
  \item The basic edition resources of the \textit{Spell} are available
    in a pop-up menu.
  \item In the meanwhile, please consider this new resource as \texttt{still
      experimental} and under development.
\end{itemize}


\subsection*{1.18.5.6 (Jun/24/2006)}
\begin{itemize}
  \item The highlighters \textit{All} and \textit{Text} were remade.
\end{itemize}


\subsection*{1.18.5.5 (May/29/2006)}
\begin{itemize}
  \item \textit{Save} and \textit{Save as} were remade. Thank you
    \texttt{John} for the good suggestion.
  \item The user can change the selection mode by clicking the mouse
    anywhere in the \textit{Main status bar}. The order of the changes
    is sensitive to:
    \begin{itemize}
      \item \texttt{Left} - corresponds to the menu
        \textit{Format/Selection mode} \texttt{bottom to top order};
      \item \texttt{Right} - correspond to \texttt{top to bottom order}.
    \end{itemize}
  \item Tinn-R now recognizes the RGUI JGR. In the meanwhile, please consider
    it as \texttt{still experimental}.
\end{itemize}


\subsection*{1.18.5.4 (May/28/2006)}
\begin{itemize}
  \item Bug(s) fixed:
    \begin{itemize}
      \item Small bugs (detected in version 1.18.5.3) were fixed.
    \end{itemize}
\end{itemize}


\subsection*{1.18.5.3 (May/27/2006)}
\begin{itemize}
  \item Bug(s) fixed:
    \begin{itemize}
      \item Small bugs (detected in prior versions) were fixed:
        \begin{itemize}
          \item Open \textit{PDF} files;
          \item Appearance of the panel \textit{Results};
          \item Icons association from the pop-up menu \textit{RCard}.
        \end{itemize}
    \end{itemize}
  \item Small changes were made to the icons.
  \item A single click mouse in the editor gutter now selects the
    entire line.
  \item Window \textit{About} was reworked.
  \item Window \textit{Credits} (previously called \textit{Information})
    was updated to reflect the lastest changes.
  \item The echo of the instructions sent to \RR{} interpreter
    (visible in \textit{R output} window) is now optional in Tinn-R.
    In order to set it up, you can mark/unmark the option
    \textit{R output echo on} in \textit{Options/Main/Application/R/General/Settings}.
    If it is marked the \RR{} output window will show the instructions
    (not for all options of \textit{send to R}) and the \RR{} output.
    Please, remember that all these resources are \texttt{still experimental}
    and not too interactive.
\end{itemize}


\subsection*{1.18.5.2 (May/22/2006)}
\begin{itemize}
  \item Bug(s) fixed:
    \begin{itemize}
      \item Small icon changes and bugs corrections.
    \end{itemize}
\end{itemize}


\subsection*{1.18.5.1 (May/20/2006)}
\begin{itemize}
  \item Bug(s) fixed:
    \begin{itemize}
      \item A small bug related with \textit{reload} was fixed. Many
        thanks to \texttt{Michael Prager} for pointing this out to us.
      \item A couple of small bugs (pointed out by users) were fixed.
    \end{itemize}
  \item Tinn-R versions 1.18.4.6 and 1.18.5.0 were considered pre-released
    (restricted to developers only).
  \item All icons were changed. Thanks to \texttt{Philippe} for the hard
    work of its selection and organization. We hope you like this new look.
  \item All syntax are now alphabetically ordered.
  \item Whenever the user chooses \textit{Save} or \textit{Save as}
    Tinn-R will to try to recognize the active syntax and the related file
    extension.
  \item The images of the files \textit{ReadMe.html}, \textit{LizesMoi.html},
    \textit{LeiaMe.html} and \textit{LeaMe.html} were changed so that they
    reflect the latest changes.
  \item Several parts of the source code were optimized.
  \item \textit{Editor options} interface was a bit reworked.
  \item The \textit{Search in files} and \textit{R output} interface were
    reworked.
  \item Menu \textit{R} was a bit reworked.
  \item Now, the user can choose which \RR{} resources, related to
    \textit{Send to R} and \textit{Controlling R}, will be visible.
    To do that, two options are available:
    \textit{Application options/R/R menu and toolbar} and a pop-up
    menu associated to \RR{} toolbar.
\end{itemize}


\subsection*{1.18.4.5 (Apr/07/2006)}
\begin{itemize}
  \item The recommendation that, under Windows XP, the user should configure the
    appearance of windows and buttons to \texttt{Classic style of Windows}
    (\texttt{not for XP style)} is no longer used. In other words,
    the old conflict among the XPMenu component that Tinn-R uses and Windows
    the XP skins was fixed.
\end{itemize}


\subsection*{1.18.4.4 (Apr/02/2006)}
\begin{itemize}
  \item The menu \textit{Help} was reworked and new options were added to it.
    The basic idea is to show how the \texttt{HTML} files are created
    by using the tool converter \texttt{txt2tags} and, the most important,
    it now enables the users \texttt{to help us with the constructions and
      corrections of these files}. So, sorry for any language mistake, but my
    native language (José Cláudio Faria) is not English and I don't want
    to annoy anyone, anymore, with English corrections. If possible,
    please help us, because these documents could be useful to many other users.
  \item Tool \textit{Computer} has now two new buttons that enables the user
    to add and remove favorite folders.
  \item Application options interface was reworked. Now it enables user
    to set how the DOS console will be shown for compilation.
  \item As a consequence, the \textit{Tools} menu and \textit{Processing}
    toolbar also have the related options.
  \item For both file compilation ('.tex' to '.dvi') or ('.tex' to '.pdf')
    whenever the \textit{Open always after compilation (option)} \texttt{is
      marked}, Tinn-R will freeze while waiting for the MikTeX compilation
    to open the compiled file, in the other way it will not. In another
    words, whenever the \textit{Open always after compilation (option)}
    \texttt{isn't checked} Tinn-R will not freeze and the user can continue
    normally his job.
  \item Tinn-R now is closed with \texttt{//Yap (Yet Another Previewer)// as
      DVI viewer}, because it is the default released with
    \htmladdnormallink{MikTeX}{http://www.miktex.org}
    distribution, and it is a very nice and fast DVI viewer.
    Tinn-R will open only a single instance of the Yap, but allowing as
    many files as the user whishes.
  \item If the user uses the function
    \textit{utils:::setWindowTitle(paste("-",getwd()))} in Rprofile.site,
    Tinn-R still recognizes Rgui in SDI mode (it was not the case in previous
    versions between 1.18.X.X and 1.18.4.3).
\end{itemize}


\subsection*{1.18.4.3 (Mar/28/2006)}
\begin{itemize}
  \item Tinn-R versions 1.18.3.1, 1.18.3.2, 1.18.4.0, 1.18.4.1 and 1.18.4.2
    were considered pre-released (restricted to developers only).
  \item The structure of the Tinn-R ini files was changed again:

    \begin{footnotesize}
      \begin{verbatim}
        ..\Tinn-R                : ReadMe.txt file;
        ..\Tinn-R\data           : dbRcard.dbf, dbRcard.dbt, dbRCard.ndx, dbRtip.dbf,
        dbRtip.dbt, dbRTip.ndx and ReadMe.txt files;
        ..\Tinn-R\ini            : Shortcuts.tinn, Tinn.ini and ReadMe.txt
        ..\Tinn-R\ini\colors     : customColors.ini and ReadMe.txt
        ..\Tinn-R\ini\syntax     : C#.ini, C++.ini, ... , XML.ini and ReadMe.txt;
        ..\Tinn-R\ini\syntax bkp : temporary syntax file and ReadMe.txt;
        ..\Tinn-R\syntax         : deplate.xml, txt2tags.xml and ReadMe.txt.
        ..\Tinn-R\temp           : temporary files.
      \end{verbatim}
    \end{footnotesize}

\end{itemize}

\begin{footnotesize}
  \begin{quotation}
    So, the old backups will not be compatible anymore with this and future
    versions. This version will recognize the basic old system configurations,
    but not all syntax preferences. Sorry for the inconvenience, but, it was
    necessary. From now on, Tinn-R will make real system (all ini files)
    backup of your settings and preferences.
  \end{quotation}
\end{footnotesize}

\begin{itemize}
  \item Two useful tools for file conversion were added:
    \htmladdnormallink{deplate}{http://deplate.sourceforge.net/index.php}
    and
    \htmladdnormallink{txt2tags}{http://txt2tags.sourceforge.net}.
    For deplate, the extension \texttt{.dlpt} was proposed, and for txt2tags
    Tinn-R recognizes \texttt{.t2t}. The basic highlighters, based on XML,
    were proposed for both. \texttt{We hope users can help us in the
      development}. Also, the interface for syntax colors preferences for
    both are different from the already known Tinn-R interface. So, please
    consider that it is working nicely, but that it will be temporary.
  \item Tinn-R now enables you to compile \texttt{LaTeX} files with
    \htmladdnormallink{MikTeX}{http://www.miktex.org} and view the
    \texttt{DVI} and \texttt{PDF} results (Yap and Acrobat) and also
    to see \texttt{HTML} files in your preferred browser starting from
    Tinn-R.
  \item Application options interface was reworked. It now enables you
    to set the basic preferences for file conversion, file compilation
    and viewers.
  \item As a consequence the \textit{Tools} menu and toolbar have new
    options to file conversion, compilations and file view.
  \item All help files were changed from '.txt' to '.html' using the
    new txt2tags tool inside Tinn-R: we hope you like the new resources.
  \item Several parts of the source code were optimized.
  \item Menu \textit{Web} was reworked.
  \item Menu \textit{Help} was reworked.
\end{itemize}


\subsection*{1.18.3.0 (Mar/16/2006)}
\begin{itemize}
  \item The structure of the Tinn-R ini files was changed:

    \begin{footnotesize}
      \begin{verbatim}
        ..\Tinn-R                : ReadMe.txt file;
        ..\Tinn-R\data           : dbRcard.dbf, dbRcard.dbt, dbRCard.ndx, dbRtip.dbf,
        dbRtip.dbt, dbRTip.ndx and ReadMe.txt files;
        ..\Tinn-R\ini            : Shortcuts.tinn, Tinn.ini and ReadMe.txt
        ..\Tinn-R\ini\colors     : customColors.ini and ReadMe.txt
        ..\Tinn-R\ini\syntax     : C#.ini, C++.ini, ... , XML.ini and ReadMe.txt;
        ..\Tinn-R\ini\syntax bkp : temporary syntax file and ReadMe.txt;
        ..\Tinn-R\temp           : temporary files.
      \end{verbatim}
    \end{footnotesize}

\end{itemize}

\begin{footnotesize}
  \begin{quotation}
    Therefore, old backups will not be compatible with this and next versions anymore.
    This version will recognizes the basic old system configurations, but not
    the syntax preferences. Sorry for the inconvenience, but they were necessary.
    From now on, Tinn-R will make real system (all ini files) backup of your
    settings and preferences.
  \end{quotation}
\end{footnotesize}

\begin{itemize}
  \item \textit{Syntax color preferences} interface was reworked and
    it has new options.
  \item Tinn-R has three multi-highlighters: Sweave, HTMLcomplex and
    PHPcomplex:
    \begin{enumerate}
      \item Sweave       = TeX  \& \RR{}
      \item HTML complex = HTML \& JavaScript
      \item PHP complex  = HTML \& JavaScript \& PHP
    \end{enumerate}
\end{itemize}

\begin{footnotesize}
  \begin{quotation}
    These highlighters have no priorities when you set the syntax color
    preferences. Thus, if you change the color preferences of any of
    these multi-highlighters (Sweave, HTML complex and PHP complex) these
    settings will be valid only in the current Tinn-R session and will
    not be saved when Tinn-R is closed. So, if you want to make permanent
    changes, set the  preferences from all simple highlighters (R, TeX,
    HTML, JavaScript and PHP).
  \end{quotation}
\end{footnotesize}

\begin{itemize}
  \item Menu \textit{Option/Syntax} was reworked.
  \item A lot of the source code was optimized.
  \item All Tinn-R color dialogs were reworked and they
    automatically save and recover all preferred colors.
\end{itemize}


\subsection*{1.18.2.1 (Mar/08/2006)}
\begin{itemize}
  \item Some aspects of \textit{Syntax color preferences} interface
    were optimized.
  \item \textit{Replace text} and \textit{Search text} interface now
    recognize if more than one text line was selected and set the
    option \textit{Selected text only} automatically.
\end{itemize}


\subsection*{1.18.2.0 (Mar/06/2006)}
\begin{itemize}
  \item Now Tinn-R has three multi-highlighters: Sweave, HTMLcomplex
    and PHPcomplex:
    \begin{enumerate}
      \item Sweave      = TeX  \& \RR{}
      \item HTMLcomplex = HTML \& JavaScript
      \item PHPcomplex  = HTML \& JavaScript \& PHP
    \end{enumerate}
\end{itemize}

\begin{footnotesize}
  \begin{quotation}
    These highlighters have no priorities when you set the
    syntax color preferences. Thus, if you change color preferences of
    any of these simple highlighters (Sweave, HTMLcomplex or PHPcomplex) these
    settings will be valid only in the current Tinn-R session and will not be
    saved when Tinn-R is closed. So, if you want to make permanent changes, set
    the preferences of all these simple highlighters (TeX, \RR, HTML, JavaScript
    and PHP) from the multi-highlighters (Sweave, HTMLcomplex and PHPcomplex)
    respectively.
  \end{quotation}
\end{footnotesize}

\begin{itemize}
  \item Several parts of the source code were optimized.
  \item \textit{Tools/Project} interface was reworked; now it has new resources.
\end{itemize}


\subsection*{1.18.1.10 (Feb/18/2006)}
\begin{itemize}
  \item Bug(s) fixed:
    \begin{itemize}
      \item A small bug with Sweave highlighter related to color preferences
        was fixed. In fact, Sweave highlighter is a multi-highlighter made
        from prior \RR{} and TeX highlighters. So, if you set the color
        of the background to all TeX elements (comment, space, etc) with
        the same color (gray for example), you can get the Sweave syntax with
        two background colors: one for TeX (gray) and another to \RR{} (white).
      \item A small bug with \textit{View/Line numbers} was fixed.
    \end{itemize}
\end{itemize}


\subsection*{1.18.1.9 (Feb/18/2006)}
\begin{itemize}
  \item Bug(s) fixed:
    \begin{itemize}
      \item A small bug with menu \textit{Edit/Undo} and \textit{Edit/Redo}
        was fixed. Many thanks to users for pointing this out.
      \item A small bug with menu \textit{Edit/Undo} and \textit{Edit/Redo}
        was fixed. Many thanks to users for pointing this out.
    \end{itemize}
  \item Sweave syntax highlighter was added to Tinn-R with .rnw and .snw
    extensions. For Tinn-R the delimiters are:

    \begin{footnotesize}
      \begin{verbatim}
        >>=  start a R block
        @    close it.
      \end{verbatim}
    \end{footnotesize}

  \item Sweave provides a flexible framework for mixing text and S code
    for automatic document generation. A single source file contains both
    documentation text and S code, which are then woven into a final
    document containing:
    \begin{enumerate}
      \item The documentation text together with
      \item The S code and/or
      \item The output of the code (text, graphs) by running the S code
        through an S engine like R.
    \end{enumerate}
\end{itemize}

\begin{footnotesize}
  \begin{quotation}
    Hence, the full power of LaTeX (for high-quality typesetting) and S (for data
    analysis) can be used simultaneously. For more information see \texttt{?Sweave}
    from \RR{}.
  \end{quotation}
\end{footnotesize}

\begin{itemize}
  \item All combo-box related with files and extensions were reworked
    and now they are simpler.
\end{itemize}


\subsection*{1.18.1.8 (Feb/12/2006)}
\begin{itemize}
  \item Bug(s) fixed:
    \begin{itemize}
      \item Small bug with \textit{Copy} of pop-up menu \textit{R card}
        was fixed.
      \item A couple of small bugs (pointed out by users) were fixed.
    \end{itemize}
  \item Main menu \textit{Help} and \textit{Web} were reworked.
  \item Tinn-R and SciViews-R GUI performance was enhanced.
  \item Small correction with \textit{R explorer} interface
    related to enabled/disabled options.
  \item Two new options were added to menus \textit{R} and \textit{R toolbar}:
    \textit{Edit} and \textit{Fix}. These options enable
    you to edit or to fix \RR{} objects (if they exist in the \RR{}
    environment).
\end{itemize}


\subsection*{1.18.1.7 (Jan/14/2006)}
\begin{itemize}
  \item Small correction with \textit{R card} and \textit{R explorer}
    interface related to enabled/disabled options.
\end{itemize}


\subsection*{1.18.1.6 (Jan/10/2006)}
\begin{itemize}
  \item Bug(s) fixed:
    \begin{itemize}
      \item Small bugs pointed out by users were fixed. Many thanks users!
    \end{itemize}
  \item A new option \textit{GUI Wiki} was added to menu
    \textit{Web/R-information on line}. This Wiki was designed mainly
    to deal with \RR{} beginners problems. Although we
    would like to emphasize the use \RR{} GUIs (Graphical User
    Interfaces), this Wiki is not restricted to those GUIs: one
    can also deal with command-line approaches. Thus, the main idea is
    to have material contributed by both beginners, and by
    more advanced \RR{} users, that will help novices or casual
    R users.
  \item In menus \textit{R} and \textit{R toolbar} the position of
    the buttons was a little changed for a more logical arrangement;
    additionally, some icons were changed and a new and very useful option
    was added: \textit{Example}.
  \item Print/preview, \RR{} explorer and \RR{} card Interfaces
    were reworked.
  \item Under TCP/IP some functions to \textit{Controlling R} were
    reworked, which are now more user friendly.
  \item Horizontal and vertical splits were deeply reworked.
    Now they are more user friendly. In consequence, the main menu \textit{View}
    was changed.
  \item The pop-up menu of the open files was enhanced and two new
    options were added: \textit{Close all} and \textit{Close others}.
    The \textit{Close others} was also added to the Main menu \textit{File}.
    The first enables you to close all files and the second, all files except
    the active (current) one. Please note that both will be enabled only if
    two or more files are open.
  \item A lot of procedures and functions were optimized.
\end{itemize}


\subsection*{1.18.1.5 (Jan/01/2006)}
\begin{itemize}
  \item A Tinn-R card, \textit{Help/Tinn-R card (PDF)}, was added
    to the project. Many thanks to \texttt{Suresh Krishna}.
  \item New interface/options for printing, which now enables you to
    print preview with new resources. In consequence, menu
    \textit{File/Print preview} was removed.
\end{itemize}

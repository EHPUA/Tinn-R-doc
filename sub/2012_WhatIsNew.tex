
\section{Versions released in 2012 (03)}
\subsection{2.4.1.5 (Dez/06/2012)}

\begin{itemize}
  \item Bug(s) fixed:
    \begin{itemize}
      \item A bug related to the \texttt{Options/Application/Rgui/Recognition (Rgui)} accidentally added
        in the prior compilation (pre-release 2.4.1.2) was fixed.
    \end{itemize}
  \item The versions 2.4.0.2, 2.4.1.0, 2.4.1.1, 2.4.1.2, 2.4.1.3 and 2.4.1.4 were restricted to pre-release testers:
    Many thanks to all testers and for the suggestions!
  \item Basic suport to \htmladdnormallink{Pandoc}{http://johnmacfarlane.net/pandoc/index.html} (a universal document converter) was added.
  \item The main menu \texttt{Help/Main/File conversion (introduction)} was updated. Now it also contains an example of \texttt{pandoc.markdown} file.
  \item The User guide was thoroughly updated.
  \item The file \texttt{Rconfigure\_default.r} was updated.
  \item The convertion tools (Deplate, Pandoc and Txt2tags) are currently enabled to all file extensions (including empty extension).
  \item The main menu \texttt{Web} was updated and has new options.
  \item The message of any problem related to \texttt{R/Configure/Pemanent (Rprofile.site)} was enhanced.
\end{itemize}

\subsection{2.4.0.1 (Nov/07/2012)}

\begin{itemize}
  \item Bug(s) fixed:
    \begin{itemize}
      \item A bug related to \texttt{CTRL + ENTER} was fixed. Many thanks to \texttt{Leandro Marino} for pointing it out!
      \item A "bug" (\textit{the Microsoft eventually makes some drastic changes}) related to the Windows 7 (64 bit)
        and the path detection of R (32-bit and 64-bit) was fixed.
    \end{itemize}
  \item The internal shortcut \texttt{CTRL + ENTER} doesn't break the current line at the cursor position any more, i.e, it preserves
    the entire line content and starts a new one. Many thanks to \texttt{Leandro Marino} for pointing it out!
  \item The versions 2.3.7.4 and 2.4.0.0 were restricted to pre-release testers: many thanks to all testers and for the suggestions!
  \item The graphical interface was updated with some improvements, mainly the \texttt{Application options}:
    \begin{itemize}
      \item A hierarchical tree view replaced the classical tabs approach. It is better for large options like now.
      \item The options related to path of DVI and PDF viewer was removed. The default system for \texttt{.dvi} and \texttt{.pdf} will be used.
      \item The option to close a prior instance of the viewer (DVI and PDF) before compilation is now independent. It gives more user control.
    \end{itemize}
  \item A new toolbar \texttt{Format} was added to the main toolbar.
  \item A new resource allowing reformat R code (selection or whole file) using \texttt{formatR} package was added. 
    The icon resource was placed in the \texttt{Format} toolbar, so that from this version on,
    the \texttt{formatR} package will be necessarily together with the already traditional \texttt{TinnR} and \texttt{svSocket} packages.
  \item Due to the new resource related to the reformat code the variable .trPaths was changed. As a result,
    it will be necessary to run again the \texttt{R/Configure/Permanent (Rprofile.site)}.
    In this case, \textbf{do not forget to remove any prior script generated by Tinn-R in the Rprofile.site file}.
  \item The \texttt{R/Hotkeys} interface was thoroughly reworked. Now it has two tabs, \texttt{Default} and \texttt{Custom}:
    \begin{itemize}
      \item \texttt{Default}: contains the already traditional instructions of Tinn-R;
      \item \texttt{Custom}: \textbf{allows the user to customize any instructions} to be send to R interpreter
        (thanks to Philemon Lenherr for the suggestion). The instructions must be as follows:
        \begin{itemize}
          \item Simple: \texttt{search()}. The R interpreter will receive \texttt{$>$ search()};
          \item Replace word or small selection: \texttt{View(\%s, title='View of iris dataset')}.
            If the editor cursor is over the word \texttt{iris} or it is selected,
            the R interpreter will receive \texttt{$>$ View(iris, title='View of iris dataset')}
          \item Replace whole file: \texttt{source(\%f, echo=TRUE, verbose=TRUE)}.
            The R interpreter will receive \texttt{$>$ source(.trPaths[4], echo=TRUE, verbose=TRUE)}.
            All rules related to send file are preserved.
        \end{itemize}
    \end{itemize}
  \item Sorry, due to thoroughly changes made to the \texttt{R/Hotkeys}, all hotkeys configured prior to this version will be lost.
    It will be necessary to reconfigure them all.
  \item The chapter \texttt{Some secrets for an efficient use} is being revised by \texttt{Ricardo Pietrobon} and soon will be completed.
    Many thanks for his hard work!
\end{itemize}

\subsection{2.3.7.3 (Set/23/2012)}

\begin{itemize}
  \item Bug(s) fixed:
    \begin{itemize}
      \item Not really a bug, but a correction related to prior versions which do not recognize
        the \texttt{Options/Application/R/Rgui/Recognition/Type} is set to \texttt{Whole} related to \texttt{R Console (32-bit)}.
    \end{itemize}
  \item This version was compiled with \textbf{Code Gear 2007 running under Windows 7}.
    Previously it was compiled under Windows \textbf{Vista} or \textbf{XP}.
    We noticed that some boring stuff were automatically corrected by simply changing the operating system.
    That is, there were bugs caused by older operating systems.
  \item Some default startup values were changed. We hope it is now better for novices.
  \item Basic support to R package Knitr was added.
  \item Minor improvements in the graphical interface.
  \item The user guide has a new chapter: \texttt{Some secrets for an efficient use}.
\end{itemize}

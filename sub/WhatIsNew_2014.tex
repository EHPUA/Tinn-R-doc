
\hypertarget{2014}{}
\section{Versions released in 2014 (05)}
\index{What is New?!2014}
\subsection*{3.0.3.3 (Feb/04/2014)}
\begin{itemize}
  \item Parts of the source code were enhanced.
  \item The \textit{IPC (Inter Process Communication) communicating Rterm and Tinn-R} was re-optimised.
   Now it its approximately \textbf{4x faster than the prior version}. \verb;\o/\o/\o/
  \item The \textit{User guide} has been revised and improved.
\end{itemize}


\section{Versions released in 2014 (04)}
\subsection*{3.0.3.2 (Jan/30/2014)}
\begin{itemize}
  \item Bug(s) fixed:
    \begin{itemize}
      \item Advanced options of the editor: \texttt{Options/Application/Editor/Advanced/Want tabs}.
       Now when tabbing (if there is a selection) $<$TAB$>$ and $<$SHIFT$>$$<$TAB$>$ act really as block indent, unindent.
       It works only inside of the more important instances of SynEdit class: \texttt{Editor} and \texttt{Rterm/Log}. \\
       Within \texttt{Rterm/IO} $<$TAB$>$ has another function: to complete.
    \end{itemize}
  \item Some options of the interface \texttt{Options/Application/Editor/Advanced} are now more understandable.
   Thanks to \texttt{Berry Boessenkool} for the suggestions!
  \item The menu \texttt{Help} has a new option: \texttt{What is new?}
\end{itemize}


\subsection*{3.0.3.1 (Jan/29/2014)}
\begin{itemize}
  \item Bug(s) fixed:
    \begin{itemize}
      \item The installer of version 3.0.3.0 related to the file \texttt{data.zip}.
       The file \texttt{data.zip} was corrupted. The installer of version 3.0.3.0 has been deleted from all servers
       and we advise users to not redistribute this version. Thanks to \texttt{Mark A.} for pointing it out!
    \end{itemize}
  \item The \textit{STOP} button is working again for \texttt{Rgui.exe} (but is not active for \texttt{Rterm.exe} yet).
\end{itemize}


\subsection*{3.0.3.0 (Jan/28/2014)}
\begin{itemize}
  \item Bug(s) fixed:
    \begin{itemize}
      \item Pop-up menu of: \texttt{Tools/Database/Completion},
       \texttt{Tools/R/Explorer} and \texttt{Tools/R/Card}.
    \end{itemize}
  \item Parts of the source code were enhanced.
  \item The \textit{IPC (Inter Process Communication) communicating Rterm and Tinn-R} was optimised.
   Now it its approximately \textbf{8x faster and also more accurate}. This really was an old dream!
  \item The \textit{User guide} has been revised.
  \item The menu \texttt{Tools/Processing/Compilation (Latex)} has a new option: Make index (makeindex).
   The default shortcut is \textit{CTRL + ALT + I}.
  \item The \texttt{Rterm} support to the \textit{function debug} and the \textit{package debug} was a bit enhanced.
   The necessary instruction (below) is automatically sent to the \RR{} interpreter from this version.

    \begin{footnotesize}
      \begin{verbatim}
        options(debug.catfile = 'stdout')
      \end{verbatim}
    \end{footnotesize}
\end{itemize}


\subsection*{3.0.2.8 (Jan/22/2014)}
\begin{itemize}
  \item Bug(s) fixed:
    \begin{itemize}
      \item \RR{} highlighter: \textit{quote} argument of \textit{read.table} function.
      \item Pop-up menu of \texttt{Tools/R/Mirrors}.
    \end{itemize}
  \item TinnRcom package was upgraded to version (1.0-15).
   The package will be automatically updated.
  \item \texttt{Tools/R/Mirrors}:
    \begin{itemize}
       \item It has a new status bar showing the default repository
       \item It has a new button on the tool bar which enables the opening of the
             URL: current and default.
    \end{itemize}
  \item The \texttt{Print preview} interface was a bit enhanced.
  \item Some icons of the \texttt{Main} interface and \texttt{Tools} panel were changed.
  \item From now \htmladdnormallink{SumatraPDF}{http://blog.kowalczyk.info/software/sumatrapdf/free-pdf-reader.html}
   will be the default viewer of the Tinn-R User guide.
   If Sumatra is the default system PDF viewer it will be used. Otherwise, a compact version
   released jointly with Tinn-R (\texttt{Install.path/sumatra/SumatraPDF.exe}) will be used.
\end{itemize}

\section{Versions released in 2014 (02)}
\subsection*{3.0.2.9 (Jan/26/2014)}
\begin{itemize}
  \item Bug(s) fixed:
    \begin{itemize}
      \item Pop-up menu of: \texttt{Tools/Database/Completion},
       \texttt{Tools/R/Explorer} and \texttt{Tools/R/Card}.
    \end{itemize}
  \item The IPC (Inter Process Communication) \texttt{communicating Rterm and Tinn-R} was optimised.
   Now it its approximately $8x$ faster and also more accurate. This really was an old dream!
  \item The \textit{User guide} has been revised.
  \item The menu \textit{Tools/Processing/Compilation (Latex)} has a new option: Make index (makeindex).
   The default shortcut is \textit{CTRL + ALT + I}.
\end{itemize}


\subsection*{3.0.2.8 (Jan/22/2014)}
\begin{itemize}
  \item Bug(s) fixed:
    \begin{itemize}
      \item \RR{} highlighter: \texttt{quote} argument of \texttt{read.table} function.
      \item Pop-up menu of \texttt{Tools/R/Mirrors}.
    \end{itemize}
  \item TinnRcom package was upgraded to version (1.0-15).
   The package will be automatically updated.
  \item \texttt{Tools/R/Mirrors}:
    \begin{itemize}
       \item It has a new status bar showing the default repository
       \item It has a new button on the tool bar which enables the opening of the
             URL: current and default.
    \end{itemize}
  \item The \texttt{Print preview} interface was a bit enhanced.
  \item Some icons of the \texttt{Main} interface and \texttt{Tools} panel were changed.
  \item From now \htmladdnormallink{SumatraPDF}{http://blog.kowalczyk.info/software/sumatrapdf/free-pdf-reader.html}
   will be the default viewer of the Tinn-R User guide.
   If Sumatra is the default system PDF viewer it will be used. Otherwise, a compact version
   released jointly with Tinn-R (\texttt{Install.path/sumatra/SumatraPDF.exe}) will be used.
\end{itemize}

\hypertarget{2005}{}
\section{Versions released in 2005 (25)}
\subsection*{1.18.1.4 (Dec/24/2005)}
\begin{itemize}
  \item The extensions '*.s' and '*.ssc' were added to \RR{} syntax.
  \item New interface/options for printing, which now enables you to
    choose the file's printing range: All, Selection and Pages.
\end{itemize}


\subsection*{1.18.1.3 (Dec/17/2005)}
\begin{itemize}
  \item The changes (since Tinn-R 1.17.2.6) have been considered stabilized.
  \item A new option \textit{R syntax as default} was added to the main
    options and menu \textit{Options}. It enables you to choose if a new
    file and file without extension will be opened for edition with \textit{R}
    or \textit{General} syntax as default.
\end{itemize}


\subsection*{1.18.1.2 beta (Dec/11/2005)}
\begin{itemize}
  \item Tinn-R version 1.18.1.1 BETA was considered pre-released
    (restricted to beta testers only). Many thanks to all testers.
  \item The pop-up menu of the \textit{R explorer} was enhanced and
    new options were added.
  \item You can now drag and drop objects from the \textit{R explorer}
    to editor.
  \item Many procedures and functions were optimized.
\end{itemize}


\subsection*{1.18.1.1 beta (Dec/03/2005)}
\begin{itemize}
  \item Bug(s) fixed:
    \begin{itemize}
      \item A small bug with the install program (related to desktop shortcut creation) was fixed.
        The old shortcut was pointing
        to \textit{Tinn-R.exe.manifest} file and not to \textit{Tinn-R.exe}.
    \end{itemize}
  \item An experimental \RR{} explorer was added to the program. This was
    accomplished by adding a new tab named \textit{R explorer} to the panel
    \textit{Tools}; it enables you to see \RR{} objects from
    Tinn-R. Meanwhile, this explorer must be updated by the user
    because it is not automatically done by the \RR{} server.
    It was based on the SciViews \RR{} explorer, but it is not exactly
    the same.
  \item The speed of the procedures/updates with the panel \textit{Tools}:
    \textit{Computer} and \textit{Project}, were enhanced.
  \item A new option \textit{Send to R} was added to the pop-up menu of
    \textit{Project}. It enables you to send the whole project,
    individual groups or individual files to the server without
    opening it for edition.
  \item A new menu option was added to the \textit{R} menu:
    \textit{Server: connections and tests}. This menu will open a
    new window of the program that allows you to connect and test
    \RR{} as server using TCP/IP and DDE protocol.
    \begin{itemize}
      \item The window \textit{Application options} was somehow reworked
        to reflect the new features of the program (specially TCP/IP
        protocol to communicate with \RR{} server). This resource is
        still experimental and, under tests, it shows instability. So,
        if any problem occurs, disconnect the server
        \textit{Server: connections and tests} and all new resources
        (using another option - DDE and/or temporary files) will work.
        In the future the use of TCP/IP protocol by Tinn-R will be
        enhanced.
    \end{itemize}
  \item An experimental code completion was added. For a basic use:
    \begin{enumerate}
      \item Set the options on \textit{Application options};
      \item Like \textit{Tip}, if you check \textit{Always shown}, after the
        data object type and '\$' related completion will be shown;
      \item Select the desired completion and press ENTER to make it work.

        \begin{footnotesize}
          \begin{verbatim}
            A tip: my preferred shortcuts are: ``CTRL + D`` for tip after //(//
            and ``CTRL + SHIFT + D`` after '$' for code completion.
          \end{verbatim}
        \end{footnotesize}

    \end{enumerate}
  \item A new menu option was added to the \textit{View} menu:
    \textit{R output}. This menu will open a new window on the
    bottom of the program that allows you to receive the output
    of the \RR{} as server using only TCP/IP protocol. It is still
    experimental.
  \item The communication with \RR{} under DDE protocol was enhanced:
    it is now faster and more stable.
  \item Tool \textit{Computer} now allows you to open (with double
    click or drag and drop) any type of files, and no only
    the ones recognized.
  \item Two new syntax highlighters were added to Tinn-R:
    \begin{enumerate}
      \item Fortran: useful to \RR{} developers;
      \item Visual basic: still lives.
    \end{enumerate}
  \item A new pop-up menu was added to the list view of Tools
    \textit{Computer} that enables you to refresh and to choose
    the view style.
  \item Many procedures and functions were optimized.
\end{itemize}


\subsection*{1.17.2.6 (Oct/27/2005)}
\begin{itemize}
  \item Bug(s) fixed:
    \begin{itemize}
      \item A small bug related to \textit{On top} was checked and Tinn-R sometimes
        minimizes and others times it doesn't (hides in the task-bar), was fixed. Many
        thanks to \texttt{Mihai Nica} for pointing that out.
    \end{itemize}
\end{itemize}


\subsection*{1.17.2.5 (Oct/25/2005)}
\begin{itemize}
  \item The memory use increases (16 k every second) related to
    some prior versions was fixed. Many thanks to users for pointing this out.
  \item The SynEdit.pas unit component was updated to the latest version
    1.428 2005/10/07 21:16:10. So, gutter width was always calculated
    using editor font was fixed.
  \item Shortcut for \textit{Format/Block/Mark} was changed from
    \texttt{CTRL + ALT + B} to \texttt{CTRL + ALT + S}.
\end{itemize}


\subsection*{1.17.2.4 (Oct/08/2005)}
\begin{itemize}
  \item Bug(s) fixed:
    \begin{itemize}
      \item A small bug with \textit{Options/Syntax/Color preferences}
        related to \textit{Cancel} option was fixed.
    \end{itemize}
  \item The \textit{TeX highlighter} was a little reworked: now it
    matches brackets. Thanks \texttt{Sheldon Kelly} for suggest it.
  \item The \textit{Restore} procedure was a little changed: it now
    checks if any file or project was changed (and not saved) prior
    to restart the program.
  \item The database was a little reworked. So, if you had any old
    Tinn-R version installed, we recommend that you restore the
    data from this version. To do this, select \textit{Tools/Restore/Database}
    from the main menu and choose the \textit{data} folder where Tinn-R
    was just installed, as a prior backup.
  \item Main program icon was a little changed (the \RR{} is now darker).
\end{itemize}


\subsection*{1.17.2.3 (Oct/03/2005)}
\begin{itemize}
  \item Bug(s) fixed:
    \begin{itemize}
      \item A small bug with database (related with version 1.17.2.2 only)
        was fixed.
    \end{itemize}
\end{itemize}


\subsection*{1.17.2.2 (Oct/02/2005)}
\begin{itemize}
  \item Bug(s) fixed:
    \begin{itemize}
      \item A bug with messages below:

        \begin{footnotesize}
          \begin{verbatim}
            "Severe problem! The program cannot be initiated. Please, contact the
            developers."

            "Access violation at address 'hexa number' in module 'Tinn-R'. Read address
            'hexa number'."

            It was now fixed. This bug was related to more than one file were selected
            from the explorer to be opened and Tinn-R was not running.
            So, if Tinn-R is not running (if and only if), avoid to make it because it
            will open one instance of Tinn-R and one file (the first in alphabetic
            order) only. BTW, if Tinn-R is running, you can select simultaneously any
            desired number of files.
          \end{verbatim}
        \end{footnotesize}

      \item A small bug when you did a global replace in Tinn-R, i.e.,
        Search/Replace/OK/Yes to all, and get the following error message:
        Assertion failure (D:$\backslash$...$\backslash$SynEdit.pas,
        line 1460), was fixed: thanks \texttt{John}!
    \end{itemize}
  \item Tinn-R versions 1.17.1.2 and 1.17.2.1 were considered pre-released
    (restricted to beta testers only). Many thanks to all testers.
  \item The picture, when opening the program, doesn't appear anymore.
  \item Main program icon was changed: I hope you like the new one.
    BTW, I got problems because some parts of the Windows showed the
    old icon and others the new. I spent a lot of time to find the
    original problem: ShellIconCache.

    \begin{footnotesize}
      \begin{verbatim}
        SO, IF YOU GET INCORRECT ICON DISPLAYED ON WINDOWS, AFTER INSTALLING THE NEW
        VERSION OF TINN-R, PLEASE, READ THE INSTRUCTIONS BELOW:

        For acceleration of the show of the icons Windows stores images in the ICON
        CACHE (ShellIconCache) a hidden icon cache file in your windows directory.

        Sometimes the icon of the object changes, but Windows shows the old icon
        instead of the former one. To solve this problem we suggest to use the
        program IconChanger (shareware available at http://www.shelllabs.com/).

        If you have just installed Tinn-R with new icon but Windows has not changed
        the image, the guide advises you to choose REBUILD ICON CACHE and if
        it will not help then choose REMOVE ICON CACHE.

        If you have chosen REBUILD the icon cache will start reconstructing at
        once. If you have chosen REMOVE, you will see the warning  message.
        Choose YES, and then you should restart your computer.
      \end{verbatim}
    \end{footnotesize}

  \item If Tinn-R is closed when maximized, when restarted, it will open
    maximized but it will remember the last position when not maximized.
  \item A new shortcut \texttt{CTRL + ENTER} was added. When pressed, Tinn-R
    sends the current line (entire) to \RR{} interpreter and adds a line break
    at cursor position.
  \item A new toolbar with Undo and Redo was added.
  \item A new button was added to the \textit{Misc toolbar} that allows
    you to choose how Tinn-R will behave after sending anything to \RR{}
    interpreter, if checked, the focus will return to Tinn-R.
  \item The function \textit{Clear all} was updated and it is now faster.
    If any problem occur with different machines and OS, please, tell us.
  \item The menu \textit{Web} was a little reworked and a new option is
    available: \textit{Web/Tinn-R} with links to home page (sciViews),
    sourceforge server and check for update (from sciViews server).
  \item A new menu option \textit{Help/Tinn-R citation} was added. It enables
    you to put the Tinn-R citation in the clipboard.
  \item After restoring the configurations file, Tinn-R now can
    close and restart itself.
\end{itemize}


\subsection*{1.17.1.1 (Aug/28/2005)}
\begin{itemize}
  \item Bug(s) fixed:
    \begin{itemize}
      \item A small bug with recent files was fixed.
    \end{itemize}
  \item Tinn-R versions 1.16.1.9 beta and 1.16.1.10 beta were considered
    pre-released (restricted to beta testers only): thanks to all testers.
  \item The SynEdit component was updated to the latest version (v2.0.1 beta).
  \item The windows \textit{Application options} and \textit{Editor options}
    were a little reworked.
  \item The main menu \textit{File} was a little reworked.
\end{itemize}


\subsection*{1.16.1.8 (Aug/07/2005)}
\begin{itemize}
  \item Bug(s) fixed:
    \begin{itemize}
      \item A small bug with restore database was fixed.
    \end{itemize}
\end{itemize}


\subsection*{1.16.1.7 (Aug/05/2005)}
\begin{itemize}
  \item Tinn-R version 1.16.1.6 beta was considered pre-released
    (restricted to beta testers only): thanks to all testers.
  \item The \textit{R highlighter} was a little reworked: identifiers
    with the minus signal '-' (ex: codes-deprecated) will not be
    recognized, any more.
  \item Tinn-R team has a new member working on the source code:
    welcome Huashan Chen.
  \item Tinn-R project (source and bin) is now also available from
    \htmladdnormallink{SourceForge}{https://sourceforge.net/projects/tinn-r}.
    Thanks \texttt{Huashan}.
  \item The incompatibility of Tinn-R with WinNT 4.0/SP6 has been solved.
    Thanks \texttt{Uwe}.
  \item In \textit{R toolbar} the position of the buttons was a little changed
    for a more logical arrangement.
  \item After installation, the \textit{Tinn-R/data} folder contains
    data.zip (102 KB). When starting Tinn-R, the files it contains
    are automatically unpacked, and for each user four files
    are generated: dbRcard.dbf, dbRcard.dbt, dbRtip.dbf and dbRtip.dbt
    plus two (dbRcard.ndx and dbRtip.ndx) index files. This way, each
    user has an independent database.
  \item Also, now each user has independent configurations files.
  \item Considering the two topics above, Tinn-R is now multiuser. So,
    for instance, you can have an administrator account with full access
    to all files for installing programs and maintaining the machine.
    But for everyday work, you can log in as \texttt{main user} or
    \texttt{normal user} (with restricted access) and run Tinn-R
    without problems in those restricted environments. This feature
    is very useful to educational purposes in statistical laboratories.
  \item Only three folders are generated by Tinn-R for each user:
    \begin{itemize}
      \item Windows XP:
        \begin{footnotesize}
          \begin{itemize}
            \item \textit{Drive:$\backslash$Documents and settings$\backslash$UserName$\backslash$Application data$\backslash$Tinn-R$\backslash$ini$\backslash$}
              with two files \textit{Tinn.ini} and \textit{Shortcuts.tinn};
            \item \textit{Drive:$\backslash$Documents and settings$\backslash$UserName$\backslash$Application data$\backslash$Tinn-R$\backslash$data$\backslash$}
              with six database files (see above);
            \item \textit{Drive:$\backslash$Documents and settings$\backslash$UserName$\backslash$Application data$\backslash$Tinn-R$\backslash$temp} for essential temp files (will be automatically removed
              by Tinn-R after each session).
          \end{itemize}
        \end{footnotesize}
      \item Windows NT:
        \begin{footnotesize}
          \begin{itemize}
            \item \textit{Driver:$\backslash$Winnt$\backslash$Profiles$\backslash$UserName$\backslash$Application data$\backslash$Tinn-R$\backslash$ini};
            \item \textit{Driver:$\backslash$Winnt$\backslash$Profiles$\backslash$UserName$\backslash$Application data$\backslash$Tinn-R$\backslash$data};
            \item \textit{Driver:$\backslash$Winnt$\backslash$Profiles$\backslash$UserName$\backslash$Application data$\backslash$Tinn-R$\backslash$temp}.
          \end{itemize}
        \end{footnotesize}
    \end{itemize}
  \item The database component (TDBF) was updated to the latest version.
  \item Menu \textit{Tools} was reworked a little and new options were
    added to backup/restore your personal database and configuration files.
\end{itemize}


\subsection*{1.16.1.5 (Jul/15/2005)}
\begin{itemize}
  \item Bug(s) fixed:
    \begin{itemize}
      \item A small bug with project (\textit{Save} and \textit{Save as}),
        related to automatic extension '.tps', was fixed.
    \end{itemize}
  \item Tinn-R now works with \RR{} in MDI mode if the device graphic is
    maximized with no more limitations of the number of device graphic,
    as pointed in the Tinn-R 1.15.1.7 (07 Apr 2005).
  \item If you use the function \textit{utils:::setWindowTitle(paste("-",getwd()))}
    in Rprofile, Tinn-R still recognizes Rgui (was not the case in
    previous versions).
  \item A new menu option (and respective button and hotkey) was added to
    the main menu: \textit{R/Controlling R/List structure of selected variable}
    that allows you to list the structure of any \RR{} object/variable.
  \item A new menu (and respective pop-up menu) was added to the main menu:
    \textit{File/Put full path in clipboard}. This option allows you to
    put the full path of the file in the clipboard with two options: Unix
    mode (../..) or Windows (..$\backslash$..) (useful to get full path
    of data files).
  \item In \textit{R toolbar} the position of the buttons were a little changed
    for a more logical arrangement.
  \item The menu \textit{Format/Block} was a little reworked.
\end{itemize}


\subsection*{1.16.1.4 (May/29/2005)}
\begin{itemize}
  \item Bug(s) fixed:
    \begin{itemize}
      \item Small bug with \textit{Send marked block} was corrected.
      \item Small bug with \textit{Search} and \textit{Search in files}
        were corrected.
    \end{itemize}
  \item A new button was added to the \textit{Misc toolbar} that allow
    you unmark all marks.
  \item Menu \textit{Format}  was reworked a little bit.
\end{itemize}


\subsection*{1.16.1.3 beta (May/22/2005)}
\begin{itemize}
  \item Tinn-R versions 1.15.1.8, 1.15.1.9, 1.15.1.10, 1.15.1.11, 1.16.1.1
    and 1.16.1.2 were considered pre-released (restricted to beta
    testers only).
  \item The Tinn-R installer now proposes to associate Tinn-R with '.Rd'
    files (the \RR{} help source code).
  \item The menu \textit{Format} was changed. There are now two new
    options:
    \begin{enumerate}
      \item \textit{Format/Block/Mark};
      \item \textit{Format/Block/Unmark}.
    \end{enumerate}
\end{itemize}

\begin{footnotesize}
  \begin{quotation}
    The first menu allows you to select a given block of code (line start and
    line end) to send it at once to the \RR{} interpreter and the last menu
    entry allows you to clear an existing marked block.
    Bookmark \texttt{0} is used to mark the beggining of a marked block and
    bookmark \texttt{1} is used to mark its end. \textit{Unmark} remove them
    only if they were marked with the \textit{Block/Mark} tool. Otherwise, they
    are treated as simple bookmarks. Note that, whatever the way you defined
    them, you can always change their position as usual.
    There is a distinct marked block defined for each open file, and the tools
    are enabled only if a block is marked for the current active document.
  \end{quotation}
\end{footnotesize}

\begin{itemize}
  \item As a consequence the \textit{R} menu and toolbar have new tools
    to send the marked block at once \textit{R/Send to R/Marked block (source)}
    or line by line \textit{R/Send to R/Marked block} to the \RR{}
    interpreter.
  \item Another new tool was also added to send the current selected text
    as a source file to \RR{} (and not line by line, as usual).
  \item The \textit{Main toolbar} was a little reworked to support the
    new mark block option.
  \item The accuracy of all functions (send to, or, control \RR{} )
    interacting with \RR{} was enhanced.
  \item The stability of the databases, respect to duplication of
    keys, was enhanced.
  \item The option \textit{Spanish} was added to the menu
    \textit{Help/From this version}. It contains:
    \begin{enumerate}
      \item Lea me.html;
      \item FAQ.txt;
      \item Palabras reconocidas.r.
    \end{enumerate}
\end{itemize}

\begin{quotation}
  The translation was made by \texttt{Jairo Cugliari}: thanks \texttt{Jairo}, very much!
\end{quotation}

\begin{itemize}
  \item A \RR{} card was added to the program.
    This was accomplished by adding a new tab
    named \textit{R card} to the panel \textit{Tools}.
    The \RR{} card was based on two \RR{} card already published:
    R/Rpad Reference Card by Tom Short and \RR{} reference card by Jonathan Baron.
    \begin{enumerate}
      \item It was made using a user-expandable database (DBase);
      \item The component used is named TDBF, it is free (open source), does
        not use BDE and it is not necessary to have the database server. The
        DB engine code is compiled right into the Tinn-R executable. TDBF is
        a native data access component for Delphi, BCB, Kylix and FreePascal.
        More information can be found
        \htmladdnormallink{here}{https://sourceforge.net/projects/tdbf/}.
    \end{enumerate}
  \item As a consequence, the \textit{R} menu was a little reworked. You
    have now two options for the databases:
    \begin{enumerate}
      \item \textit{R/Database/Tip} for tip management;
      \item \textit{R/Database/Card} for \RR{} card management.
    \end{enumerate}
  \item A freeware resource provided by \texttt{Angus Johnson} was added
    to the program. It is a very useful Generic Diff Format (GDIFF)
    named TextDiff. The GDIFF format can be used to efficiently describe
    the differences between two arbitrary files or folders. The format
    does not make any assumptions about the type or contents of the files,
    and thus can be used to describe the differences between text files
    as well as binary files. The work was made adapting the sources code
    of the demo project of the component to the Tinn-R project: thanks
    \texttt{Angus}, very much!
  \item As a consequence, the \textit{Tools} menu has a new tool:
    \textit{Tools/Differences}.
  \item Tinn-R is now more flexible if the screen is split: the
    functions \textit{Send to R} and \textit{Controlling R} works for
    both (top or below, left or right) frames.
  \item The \textit{Ascii chart} is now more flexible.
  \item The menu \textit{Format/Selection mode} was reworked, new
    and useful shortcuts were added (see in
    \textit{Main/Editor/Keystrokes}): thanks \texttt{Zoltan Butt}.
  \item A new field (third field from left) was added to the
    status bar with three possible values corresponding to the
    current selection state of the editor:
    \begin{enumerate}
      \item smNormal : selection in normal mode;
      \item smLine   : selection in line mode;
      \item smColumn : selection in column mode.
    \end{enumerate}
  \item The files \textit{Tinn-R\_Read me.html},
    \textit{Tinn-R\_Leia me.html} (in Portuguese) and
    \textit{Tinn-R\_Lisez moi.html} (in French) in the
    \textit{$\backslash$doc} subdirectory of Tinn-R were updated.
  \item Seven new syntax highlighters were added to Tinn-R:
    \begin{footnotesize}
      \begin{enumerate}
        \item Rd files (based on the existing TeX highlighter);
        \item Tcl/Tk;
        \item Ruby;
        \item TeX;
        \item Python;
        \item Bat;
        \item HP48.
      \end{enumerate}
    \end{footnotesize}
  \item The picture, when opening the program, was changed.
    Tinn-R program is searching for a new logo/image identity.
    This "tinny" (two n intentionally), but colorful
    bird, is a good symbol for the "tinny" Tinn-R, but rich
    in nice features regarding the edition of \RR{} code!
  \item The program user interface was reworked a little bit.
  \item Two hotkeys were changed:

    \begin{footnotesize}
      \begin{verbatim}
        <ALT+LEFT>  : Tools align left;
        <ALT+RIGHT> : Tools align Right.
      \end{verbatim}
    \end{footnotesize}

  \item The automatic extensions for all \textit{Save} action was
    improved. Now, if you select \textit{All files (*.*)} you can
    save the file with any desired extension or even with no
    extension at all. Thanks \texttt{Posta Giovanni}.
  \item The structure of the \textit{Tip database} was improved.
    The tip is not limited to 254 characters anymore, because it is
    now of \textit{memo} type.
\end{itemize}


\subsection*{1.15.1.7 (Apr/07/2005)}
\begin{itemize}
  \item Bug(s) fixed:
    \begin{itemize}
      \item Minor bug with \textit{View/Toolbars/Macro} was corrected.
    \end{itemize}
  \item Tinn-R now works with \RR{} in MDI mode if the device graphic is maximized:
    \begin{enumerate}
      \item You can work with 1..11 device graphic \textit{ACTIVE} maximized;
      \item You can work with 1..10 device graphic \textit{inactive} maximized.
        \begin{footnotesize}
          \begin{verbatim}
            The caption of \RR{} can be:
            RGui - [R Graphics: Device 1..11 (ACTIVE)] or
            RGui - [R Graphics: Device 1..10 (inactive)]
          \end{verbatim}
        \end{footnotesize}
    \end{enumerate}
  \item The state of the Caps Lock key (keyboard) doesn't influence any more the \textit{Send to R} and
    \textit{Controlling R} functions.
  \item \textit{Tools/Project} interface was improved with the replacement of the combo-box by a new dropdown button.
  \item A new toolbar named \textit{Tools} with two buttons (\textit{Toggle tools visible} and
    \textit{Align tools right/left}) was added to the main toolbar, in consequence the pop-up \textit{Tools} was removed.
  \item \textit{Project/Recent} is now checking for changes in current project.
  \item The \textit{View/Tabs} icon was changed.
  \item The delay for DDE communication with \RR{} (for call tips) is now user-selectable in the
    \textit{Options/Main/Application} dialog box (you need to load svIDE package from SciViews bundle to use this feature).
    If Tinn-R seems to freeze and you got no call tip, just increase the delay.
  \item Menu \textit{Web} and \textit{Help} were a little reworked and new information in French and Portuguese were added.
\end{itemize}


\subsection*{1.15.1.6 beta (Mar/19/2005)}
\begin{itemize}
  \item Open files in Tinn-R with \textit{Tools panel} was reworked:
    \begin{enumerate}
      \item A file must have no extension or one that Tinn-R recognizes;
      \item If the file is not opened there, you can open a copy by double clicking or dragging it into the main form;
      \item With file already opened, you can now open a copy by double clicking or dragging it in the editor area
        in Tinn-R;
      \item You can drag a project (all files of the project will be opened), a group (all files of the group will
        be opened) or an individual file.
    \end{enumerate}
  \item A new pop-up menu is available for \textit{Tools}. This menu is related with visibility and position
    of the \textit{Tools panel}.
  \item The menu \textit{Options} and \textit{View} were reworked.
\end{itemize}


\subsection*{1.15.1.5 beta (Mar/17/2005)}
\begin{itemize}
  \item \textit{Project} interface was a little reworked and two new
    options were added: \textit{Expand all}  and \textit{Collapse all}
    groups.
\end{itemize}


\subsection*{1.15.1.4 beta pre-release (Mar/12/2005)}
\begin{itemize}
  \item \textit{Application options} was reworked and an option \textit{Delay for synchronization} was added.
    Because DDE (Dynamic Data Exchange) through tcltk and svIDE packages consumes a certain time - that is
    variable between different computers - this option allows the user to customize the Tinn-R delay with the
    \RR{} as server with call-tip.
    So, it is very important to adjust this delay until getting high functionality and performance.
  \item \textit{Project} interface was reworked:
    \begin{enumerate}
      \item Only the file names are showen (no more full paths);
      \item All nodes of the project can be dragged:
        \begin{enumerate}
          \item If you drag a project all file will be opened;
          \item If you drag a group all files of this group will be opened;
          \item if you drag a single file it will be opened.
        \end{enumerate}
    \end{enumerate}
  \item The \textit{Search results} interface was reworked and two old bugs (all versions of Tinn-R that I don't knew)
    were fixed.
  \item In \textit{Tools panel/Computer} a double click on the Tinn-R project files (*.tps) will open the project
    interface. On the other hand, if you drag this file, the context of the project file will be opened
    for manual edition.
\end{itemize}


\subsection*{1.15.1.3 beta pre-release (Mar/06/2005)}
\begin{itemize}
  \item Changes to projects are monitored now.
  \item To open files in Tinn-R with \textit{Tools panel}:
    \begin{enumerate}
      \item A file must have no extension or one that Tinn-R recognizes;
      \item If the file is not opened there, you still can drag it into the main form;
      \item If any file is already opened, you can open a copy by double clicking or dragging it
        to the page control or to the main menu.
    \end{enumerate}
  \item The \RR{} highlight dictionary was updated and about 270 new functions were added to be
    compatible with \RR{} parameters completion proposal (RPCP) database.
\end{itemize}


\subsection*{1.15.1.2 beta pre-release (Mar/05/2005)}
\begin{itemize}
  \item Bug(s) fixed:
    \begin{itemize}
      \item Small (and old) bug with painted symbols '(', '[', '\{' inside the gutter was corrected: thanks \texttt{Marco}!
    \end{itemize}
  \item A new folder named \textit{res} with a single file (Tinn-R\_img.bmp) was added to the Tinn-R program.
  \item The Tinn-R binary is about 350 Kb smaller than version 1.15.1.1.
  \item The \textit{About} box was reworked.
  \item The project combo box was reworked. It is not possible to drag it any more.
  \item \textit{RegEx} was reworked: it is now possible to paste clipboard and use carriage return in it.
  \item It is now possible to place the \textit{Tools} panel at right or left of the main window.
  \item Two new hotkeys were added (not user-configurable):

    \begin{footnotesize}
      \begin{verbatim}
        <CTRL + ALT + LEFT>  : Tools align left;
        <CTRL + ALT + RIGHT> : Tools align Right.
      \end{verbatim}
    \end{footnotesize}

  \item It is possible to dock the \RR{} toolbar only at the opposite
    side of the \textit{Tools} panel, not on the same side.
\end{itemize}


\subsection*{1.15.1.1 beta pre-release (Mar/01/2005)}
\begin{itemize}
  \item A picture was added when opening the program. Thanks \texttt{Carolina} (my daughter) for basic art creation.
  \item A new file was added to the project: Tinn-R\_shortcuts.txt. It is possible to load the
    file from \textit{Help/Tinn-R/Shortcuts}.
  \item \textit{Open dialog} for Open file and \textit{Add file} to project now allow the selection of multiple files.
  \item The database was reworked and a new information (location) was added.
  \item Tab order of the search/replace interface was reworked.
  \item Due to the growth of the code source, it is not possible any more to maintain Tinn-R code synchronized 
    with the original Tinn project. So, Tinn-R is now a new open source project. 
    The version numbers is thus changed. The new convention is: AA.BB.CC.DD S/B

    \begin{footnotesize}
      \begin{verbatim}
        That is,  AA : major version
        BB : minor version
        CC : release
        DD : build
        more : beta or pre-release version
      \end{verbatim}
    \end{footnotesize}

  \item The status bar was reworked.
  \item A new menu \textit{Web} was added.
  \item The \textit{About} dialog box was reworked.
  \item The menu \textit{Project} was completely reworked.
  \item New panel named \textit{Tools} (with two tabs: \textit{Computer} and \textit{Project}) was added.
    This feature will grow up to enhance the interaction with \RR{} interpreter
    (progressive inclusion of SciViews-R dock Window feature like the object explorer).
  \item A new hotkey (not user-configurable) was added: \texttt{CTRL + /} (divide on the numeric keypad)
    show/hide the \textit{Tools} panel.
  \item Second version of the \RR{} parameters completion proposal (call-tip).
    It now works with a user-expandable database (DBase) and with a
    \RR{} server (DDE - Dynamic Data Exchange through tcltk and svIDE packages).
    That is, this version communicates with \RR{} to get \RR{} function arguments directly
    if a function is not defined in the database.
    We thus have: database (priority 1) and \RR{} server (priority 2).
    To use the \RR{} server, the package svIDE (SciViews bundle) must be loaded in the current \RR{} session.
\end{itemize}


\subsection*{0.0.9.4 r1.15 beta 2 (Feb/11/2005)}
\begin{itemize}
  \item Bug(s) fixed:
    \begin{itemize}
      \item Small bug with \texttt{CTRL + Y} (delete entire line),
        \texttt{CTRL + T} (delete word right from cursor) and \texttt{CTRL + SHIFT + Y}
        (delete words from cursor to the end of line) that did not enable the save options, was corrected.
    \end{itemize}
  \item The position of the cursor is now preserved when using \texttt{CTRL + *} (multiply on the numeric keypad)
    to insert/replace text with parameters of the current \RR{} function.
  \item Send line now works when the cursor is on the end the line.
\end{itemize}

\subsection*{0.0.9.3 r1.15 beta 1 (Feb/11/2005)}
\begin{itemize}
  \item The problem with Tinn-R 0.0.9.3 r1.15 beta 1 (released 30 Jan 2005) that stopped working due to expiration
    of trial period for Delphi was solved.
  \item The first version with the \RR{} parameters completion proposal (Call-tip):
    \begin{enumerate}
      \item It was made with a user-expandable database (DBase), that is, this version does not communicate with
        \RR{} to guess \RR{} function arguments;
      \item The component used is named TDBF, it is free (open source), does not use BDE and it is not necessary
        to have the database server. The DB engine code is compiled right into the Tinn-R executable.
        TDBF is a native data access component for Delphi, BCB, Kylix and FreePascal.
        More information can be found \htmladdnormallink{here}{https://sourceforge.net/projects/tdbf/}.
    \end{enumerate}
  \item A new hotkey (not user-configurable): \texttt{CTRL + *} (multiply on the numeric keypad) inserts/replaces
    text with parameters of the active function (a call-tip must be visible).
  \item The \textit{Application options} was reworked. It is now possible to set the basic
    application options for \RR{} Call-tip.
\end{itemize}

\subsection*{0.0.9.3 r1.14 (Feb/11/2005)}
\begin{itemize}
  \item Bug(s) fixed:
    \begin{itemize}
      \item Some minor bugs were corrected and the buttons appearance was a little modified.
      \item A bug with menu \textit{File/Recent files} not showing at startup was fixed.
      \item A bug which added the file extension twice was fixed.
        This bug was related to full name including the extension, like \textit{myfile.r}, 
        where Tinn-R recorded it as \textit{myfile.r.r}.
      \item A bug with save was fixed.
      \item A bug with Pascal files extensions was fixed.
      \item A bug with shortcut of the menu \textit{Edit/Select all} was fixed.
    \end{itemize}
  \item The problem with Tinn-R 0.0.9.3 r1.14 (released 30 Jan 2005) that stopped working due to expiration of 
    trial period for Delphi was solved.
  \item New interface/options for printing, it remembers latest choices/preferences.
  \item \textit{About} dialog box was reworked.
  \item The program user interface was a little reworked.
  \item Most of the interactive \RR{} tools do no require any more that files be saved with a valid 
    S language extension (.r or .q).
  \item The \RR{} highlight dictionary was updated:
    \begin{enumerate}
      \item About five hundred words were added;
      \item Fixed a bug with '\_' in keywords like \textit{.decode\_package\_version}.
    \end{enumerate}
  \item This version was compiled with Delphi 7 (Delphi 5 was used for previous versions).
  \item With Delphi 7, the default dialogs were re-established. So, psvDialogs is not used any more.
  \item Menu \textit{Format/Block/Uncomment/Firsts occurrence} was renamed to Uncomment First Occurrence.
  \item Add menu \textit{Help/Tinn-R/Changes on Line}.
\end{itemize}

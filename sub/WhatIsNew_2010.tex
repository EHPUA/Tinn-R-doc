
\hypertarget{2010}{}
\section{Versions released in 2010 (10)}
\subsection*{2.3.7.1 (Nov/24/2010)}
\begin{itemize}
  \item Bug(s) fixed:
    \begin{itemize}
      \item A bug, accidentally introduced in version 2.3.7.0, which averted opening a file by \texttt{Enter or Double click},
        a recognized Tinn-R file from Windows environment, was fixed.
    \end{itemize}
\end{itemize}


\subsection*{2.3.7.0 (Nov/22/2010)}
\begin{itemize}
  \item Bug(s) fixed:
    \begin{itemize}
      \item A bug related to the intermittent loss of connection (or appearent freeze) occurring on Rgui.exe was fixed.
    \end{itemize}
  \item The versions 2.3.6.4, 2.3.6.5, 2.3.6.6 and 2.3.6.7 restricted to pre-release testers.
  \item The \texttt{Application options} interface was a bit changed:
    \begin{itemize}
      \item The \texttt{Application options/R/Rterm} was split in two tabs: \texttt{Error} and \texttt{Options}.
        The tab Error has a new option: \texttt{Trying to find code errors (at the editor)*}.
        It enables the user to set Tinn-R in order to find code errors at the editor when sending instructions to Rterm.
        It may happen that the error will not be found at the right place, for example the error might be the same word appearing
        in a comment which comes before the actual along the code. In that case the user should use the shortcut \texttt{F3 (Find again)}.
        The word will appear selected, than just press \textit{OK} until finding the right error.
        The first search done internally by Tinn-R has \textit{Case sensitive} and \textit{Whole word only} as default, but,
        this is not passed to the search interface, therefore the user should just select them if convenient.
        If the error has number among letters \textit{Whole word only} is not a good option.
    \end{itemize}
  \item This version is \texttt{fully compatible with Windows 7 and R 2.12.0}.
  \item The component XPmenu was removed from the project. Windows XP users might find the Tinn-R appearance less attractive,
    but the applicative is now more stable. As soon as possible, the project will get a better option for skins.
  \item Parts of the source code were optimized.
\end{itemize}


\subsection*{2.3.6.3 (Nov/14/2010)}
\begin{itemize}
  \item Bug(s) fixed:
    \begin{itemize}
      \item A small bug (introduced in version 2.3.6.0) related to interactive use of Rterm interface,
        when R returns only one value, was fixed.
    \end{itemize}
  \item Parts of the source code were optimized.
\end{itemize}


\subsection*{2.3.6.2 (Nov/13/2010)}
\begin{itemize}
  \item Bug(s) fixed:
    \begin{itemize}
      \item A bug (introduced in version 2.3.6.1) related to automatic recognition (and setting) of the paths
        of Rterm and Rgui from R version 2.12.0 was fixed.
    \end{itemize}
  \item Parts of the source code were optimized.
\end{itemize}


\subsection*{2.3.6.1 (Nov/12/2010)}
\begin{itemize}
  \item Bug(s) fixed:
    \begin{itemize}
      \item Permanent configuration of the file Rprofile.site from the menu option: \texttt{R/Configure/Permanent (Rprofile.site)}.
    \end{itemize}
  \item Parts of the source code were optimized.
\end{itemize}


\subsection*{2.3.6.0 (Nov/10/2010)}
\begin{itemize}
  \item Bug(s) fixed:
    \begin{itemize}
      \item Paths of Rterm and Rgui from R version 2.12.0 were fixed.
    \end{itemize}
  \item The \texttt{Application options/R/Path} has a new option: \texttt{Architecture (bit)}.
    It enables the user to set R in order to run Rterm.exe or Rgui.exe in either 32 or 64 bit.
    They run independently one from the other. If the above dialog option \texttt{Use latest version (always)} is set 
    to \textbf{No}, the dialog box \texttt{Architecture (bit)} becomes disabled.
    In other words, the path must be manually set. In this case, the user preference (if valid) will not be changed when Tinn-R is starting.
    If the dialog \texttt{Use latest version (always)} is set to \textbf{Yes}, the path of the latest R version (32 or 64 bit) will found automatically.
    In case the user's computer is 64 bit and it has the latest R version installed with both options (32 or 64 bit) the latest one has the preference.
  \item Rterm interface now recognizes the single occurrence of a string pattern when using \texttt{TAB} in the prompt.
    The general behavior is not as friendly as in Rgui or in Linux console, but it is working.
  \item Parts of the source code were optimized.
\end{itemize}


\subsection*{2.3.5.2 (Apr/11/2010)}
\begin{itemize}
  \item The \texttt{User guide} was revised by \texttt{Ricardo Pietrobon}
    in an attempt to improve its flow and style. This is a work in progress,
    and so we should be improving it over time!
\end{itemize}


\subsection*{2.3.5.1 (Mar/28/2010)}
\begin{itemize}
  \item Version restricted to developers.
  \item The \texttt{User guide} was revised by \texttt{Ricardo Pietrobon} to
    improve its style and readability. So far, we have partial revisions
    implemented to chapters \texttt{Overview, Basics, Working with, Menu
      description and parts of What is new}.
  \item The \texttt{Application options/R/Basic} has a new option:
    \texttt{Smart (all send)}. When this option is set to \texttt{Yes}, all
    single line commands will not be sent through \texttt{source(...)}, but
    instead as \texttt{Send line}.
\end{itemize}


\subsection*{2.3.5.0 (Mar/04/2010)}
\begin{itemize}
  \item Bug(s) fixed:
    \begin{itemize}
      \item Small corretions in the \textit{Application options} interface.
    \end{itemize}
  \item Parts of the source code were optimized.
  \item The Rterm interface has now a simple toolbar including the more usual
    options.
  \item The \texttt{User guide} is being revised by \texttt{Ricardo Pietrobon}
    and soon will be more readable and intelligible. Up to now only chapters
    \texttt{Overview and Basics} have been worked out. Many thanks for his
    hard work!
\end{itemize}


\subsection*{2.3.4.4 (Jan/10/2010)}
\begin{itemize}
  \item Bug(s) fixed:
    \begin{itemize}
      \item A bug related to the visibility of the buttons \textit{Send contiguous} was fixed.
    \end{itemize}
\end{itemize}
